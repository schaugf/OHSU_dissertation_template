Why use \LaTeX?
Great writing deserves to be presented beautifully, and whether you think it fair or not, first impressions and aesthetics do matter when presenting important work.
Text should be evenly spaced and correctly wrapped around interjecting figures and tables, mathematical equations must be clean and consistent, and references should be so simple to use that they don't distract from the writing process.
The vast majority of modern publishers type-set their works in LaTeX, and for good reason: the system is designed from the ground up to make writing look visually appealing, consistent, and precise.
Though Microsoft Word is a fine tool for general use document preparation, LaTeX is for science and publishing. 
Even if you, the author of a scientific paper, have submitted a document to a publisher in a Word format, that publisher type-sets your writing prior to printing. 
In fact, for many reasons, the LaTeX  document engine has been likened to the Gutenburg Printing Press as an essential tool of the modern publisher.\footnote{\url{http://www.practicallyefficient.com/2017/10/13/from-boiling-lead-and-black-art.html}}

The obvious and most important distinction between a document type-set with LaTeX and a general text renderer like Word is that LaTeX compiles documents in a manner similar to the way a compiler builds software programs: it optimizes the document at every step to make sure lines are even, text is distributed, and formatting is obeyed uncompromisingly.
For example, notice how consistent the block-justified text is in this document. 
While Microsoft Word's block justification algorithms format each line independently, rendering horrendously inconsistent spacing between words within the same paragraph, LaTeX solves this problem by optimally spacing not just words on a single line, but every element in the entire document to generate a consistent look and feel.
Magic!

This document is intended to serve as a skeleton for a PhD dissertation presented to Oregon Health \& Science University, though the basic structure can easily be adapted to different institutions.
Although intimidating at first, the vast majority of regular LaTeX usage is simple and intuitive. 
Although LaTeX makes liberal use of style-files typically designed by the publisher, this example uses a basic style template included in the LaTeX software, so significant modification should not be necessary.

Here are a few helpful points to keep in mind. 
When writing LaTeX documents, write one sentence on each line in the \texttt{.tex} files; don't worry, they wrap around in the editor. 
Similarly, Paragraph breaks are designated by skipping a line.
Use an integrated editor/viewer, such as TexMaker (free and open-source, of course).
References and bibliographies effectively take care of themselves, but you need to use a reliable citation manager. Mendeley is a great, free choice.
Organize the skeleton of your document in \texttt{main.tex}, and keep separate directories for each chapter and figures. 
Staying organized can make life even better!

On a personal note, I forced myself to learn LaTeX syntax after I turned in my Master's thesis several years ago to another institution. 
At that time, I was fed up with Word's drag-and-drop formatting implosions, unreliable equation editor, and messy citation manager, so my adviser mercifully introduced me to the LaTeX system; my life's relationships with writing documents has been much better ever since.
I admit I am something of a LaTeX evangelical fan-boy, and secretly harbor a sense of superiority whenever I glance at someone else's important document and know immediately that it was prepared in Word based solely on how unpleasant it looks; regrettably, I make no apologies for myself. 
Whether you harbor a sense of superiority for your document or not, I wish you confidence in your published material made possible through this tool.

This document is a minimal example with simple illustrations of how to embed images, tables, and references into your own document. 
If you want to do something fancy, or you require a more detailed explanations of how LaTeX works under the hood, consult your nearest Google search bar.
The contents herein are provided free of charge, but as always, you get what you pay for. \\
\hspace*{\fill} Happy type-setting!
\vspace{0.5in}

\hspace*{\fill} Geoffrey F. Schau \\
\hspace*{\fill} May 2020

